% Options for packages loaded elsewhere
\PassOptionsToPackage{unicode}{hyperref}
\PassOptionsToPackage{hyphens}{url}
%
\documentclass[
]{article}
\usepackage{amsmath,amssymb}
\usepackage{iftex}
\ifPDFTeX
  \usepackage[T1]{fontenc}
  \usepackage[utf8]{inputenc}
  \usepackage{textcomp} % provide euro and other symbols
\else % if luatex or xetex
  \usepackage{unicode-math} % this also loads fontspec
  \defaultfontfeatures{Scale=MatchLowercase}
  \defaultfontfeatures[\rmfamily]{Ligatures=TeX,Scale=1}
\fi
\usepackage{lmodern}
\ifPDFTeX\else
  % xetex/luatex font selection
\fi
% Use upquote if available, for straight quotes in verbatim environments
\IfFileExists{upquote.sty}{\usepackage{upquote}}{}
\IfFileExists{microtype.sty}{% use microtype if available
  \usepackage[]{microtype}
  \UseMicrotypeSet[protrusion]{basicmath} % disable protrusion for tt fonts
}{}
\makeatletter
\@ifundefined{KOMAClassName}{% if non-KOMA class
  \IfFileExists{parskip.sty}{%
    \usepackage{parskip}
  }{% else
    \setlength{\parindent}{0pt}
    \setlength{\parskip}{6pt plus 2pt minus 1pt}}
}{% if KOMA class
  \KOMAoptions{parskip=half}}
\makeatother
\usepackage{xcolor}
\usepackage[margin=1in]{geometry}
\usepackage{color}
\usepackage{fancyvrb}
\newcommand{\VerbBar}{|}
\newcommand{\VERB}{\Verb[commandchars=\\\{\}]}
\DefineVerbatimEnvironment{Highlighting}{Verbatim}{commandchars=\\\{\}}
% Add ',fontsize=\small' for more characters per line
\usepackage{framed}
\definecolor{shadecolor}{RGB}{248,248,248}
\newenvironment{Shaded}{\begin{snugshade}}{\end{snugshade}}
\newcommand{\AlertTok}[1]{\textcolor[rgb]{0.94,0.16,0.16}{#1}}
\newcommand{\AnnotationTok}[1]{\textcolor[rgb]{0.56,0.35,0.01}{\textbf{\textit{#1}}}}
\newcommand{\AttributeTok}[1]{\textcolor[rgb]{0.13,0.29,0.53}{#1}}
\newcommand{\BaseNTok}[1]{\textcolor[rgb]{0.00,0.00,0.81}{#1}}
\newcommand{\BuiltInTok}[1]{#1}
\newcommand{\CharTok}[1]{\textcolor[rgb]{0.31,0.60,0.02}{#1}}
\newcommand{\CommentTok}[1]{\textcolor[rgb]{0.56,0.35,0.01}{\textit{#1}}}
\newcommand{\CommentVarTok}[1]{\textcolor[rgb]{0.56,0.35,0.01}{\textbf{\textit{#1}}}}
\newcommand{\ConstantTok}[1]{\textcolor[rgb]{0.56,0.35,0.01}{#1}}
\newcommand{\ControlFlowTok}[1]{\textcolor[rgb]{0.13,0.29,0.53}{\textbf{#1}}}
\newcommand{\DataTypeTok}[1]{\textcolor[rgb]{0.13,0.29,0.53}{#1}}
\newcommand{\DecValTok}[1]{\textcolor[rgb]{0.00,0.00,0.81}{#1}}
\newcommand{\DocumentationTok}[1]{\textcolor[rgb]{0.56,0.35,0.01}{\textbf{\textit{#1}}}}
\newcommand{\ErrorTok}[1]{\textcolor[rgb]{0.64,0.00,0.00}{\textbf{#1}}}
\newcommand{\ExtensionTok}[1]{#1}
\newcommand{\FloatTok}[1]{\textcolor[rgb]{0.00,0.00,0.81}{#1}}
\newcommand{\FunctionTok}[1]{\textcolor[rgb]{0.13,0.29,0.53}{\textbf{#1}}}
\newcommand{\ImportTok}[1]{#1}
\newcommand{\InformationTok}[1]{\textcolor[rgb]{0.56,0.35,0.01}{\textbf{\textit{#1}}}}
\newcommand{\KeywordTok}[1]{\textcolor[rgb]{0.13,0.29,0.53}{\textbf{#1}}}
\newcommand{\NormalTok}[1]{#1}
\newcommand{\OperatorTok}[1]{\textcolor[rgb]{0.81,0.36,0.00}{\textbf{#1}}}
\newcommand{\OtherTok}[1]{\textcolor[rgb]{0.56,0.35,0.01}{#1}}
\newcommand{\PreprocessorTok}[1]{\textcolor[rgb]{0.56,0.35,0.01}{\textit{#1}}}
\newcommand{\RegionMarkerTok}[1]{#1}
\newcommand{\SpecialCharTok}[1]{\textcolor[rgb]{0.81,0.36,0.00}{\textbf{#1}}}
\newcommand{\SpecialStringTok}[1]{\textcolor[rgb]{0.31,0.60,0.02}{#1}}
\newcommand{\StringTok}[1]{\textcolor[rgb]{0.31,0.60,0.02}{#1}}
\newcommand{\VariableTok}[1]{\textcolor[rgb]{0.00,0.00,0.00}{#1}}
\newcommand{\VerbatimStringTok}[1]{\textcolor[rgb]{0.31,0.60,0.02}{#1}}
\newcommand{\WarningTok}[1]{\textcolor[rgb]{0.56,0.35,0.01}{\textbf{\textit{#1}}}}
\usepackage{longtable,booktabs,array}
\usepackage{calc} % for calculating minipage widths
% Correct order of tables after \paragraph or \subparagraph
\usepackage{etoolbox}
\makeatletter
\patchcmd\longtable{\par}{\if@noskipsec\mbox{}\fi\par}{}{}
\makeatother
% Allow footnotes in longtable head/foot
\IfFileExists{footnotehyper.sty}{\usepackage{footnotehyper}}{\usepackage{footnote}}
\makesavenoteenv{longtable}
\usepackage{graphicx}
\makeatletter
\def\maxwidth{\ifdim\Gin@nat@width>\linewidth\linewidth\else\Gin@nat@width\fi}
\def\maxheight{\ifdim\Gin@nat@height>\textheight\textheight\else\Gin@nat@height\fi}
\makeatother
% Scale images if necessary, so that they will not overflow the page
% margins by default, and it is still possible to overwrite the defaults
% using explicit options in \includegraphics[width, height, ...]{}
\setkeys{Gin}{width=\maxwidth,height=\maxheight,keepaspectratio}
% Set default figure placement to htbp
\makeatletter
\def\fps@figure{htbp}
\makeatother
\setlength{\emergencystretch}{3em} % prevent overfull lines
\providecommand{\tightlist}{%
  \setlength{\itemsep}{0pt}\setlength{\parskip}{0pt}}
\setcounter{secnumdepth}{5}
\ifLuaTeX
  \usepackage{selnolig}  % disable illegal ligatures
\fi
\IfFileExists{bookmark.sty}{\usepackage{bookmark}}{\usepackage{hyperref}}
\IfFileExists{xurl.sty}{\usepackage{xurl}}{} % add URL line breaks if available
\urlstyle{same}
\hypersetup{
  pdftitle={Assignment 5},
  pdfauthor={Emanuele Coradin},
  hidelinks,
  pdfcreator={LaTeX via pandoc}}

\title{Assignment 5}
\author{Emanuele Coradin}
\date{2024-06-03}

\begin{document}
\maketitle

{
\setcounter{tocdepth}{2}
\tableofcontents
}
\begin{Shaded}
\begin{Highlighting}[]
\CommentTok{\#{-}{-}{-}{-}{-}{-}{-}{-}{-}{-}{-}{-}{-} Useful functions {-}{-}{-}{-}{-}{-}{-}{-}{-}{-}{-}{-}{-}}
\NormalTok{mean\_pdf   }\OtherTok{\textless{}{-}} \ControlFlowTok{function}\NormalTok{(f, lower, upper)\{}\FunctionTok{integrate}\NormalTok{(}\ControlFlowTok{function}\NormalTok{(x) x}\SpecialCharTok{*}\FunctionTok{f}\NormalTok{(x), lower, upper,}\AttributeTok{stop.on.error =} \ConstantTok{FALSE}\NormalTok{)}\SpecialCharTok{$}\NormalTok{value\}}
\NormalTok{std\_pdf }\OtherTok{\textless{}{-}} \ControlFlowTok{function}\NormalTok{(f, lower, upper) \{}
\NormalTok{  mu }\OtherTok{\textless{}{-}} \FunctionTok{mean\_pdf}\NormalTok{(f, lower, upper)}
  \FunctionTok{sqrt}\NormalTok{(}\FunctionTok{integrate}\NormalTok{(}\ControlFlowTok{function}\NormalTok{(x) (x }\SpecialCharTok{{-}}\NormalTok{ mu)}\SpecialCharTok{\^{}}\DecValTok{2} \SpecialCharTok{*} \FunctionTok{f}\NormalTok{(x), lower, upper, }\AttributeTok{stop.on.error =} \ConstantTok{FALSE}\NormalTok{)}\SpecialCharTok{$}\NormalTok{value }\SpecialCharTok{/} \FunctionTok{integrate}\NormalTok{(f, lower, upper, }\AttributeTok{stop.on.error =} \ConstantTok{FALSE}\NormalTok{)}\SpecialCharTok{$}\NormalTok{value)}
\NormalTok{\}}
\NormalTok{cumulative }\OtherTok{\textless{}{-}} \ControlFlowTok{function}\NormalTok{(f, lower, X)\{}\FunctionTok{integrate}\NormalTok{(f, lower, X,}\AttributeTok{stop.on.error =} \ConstantTok{FALSE}\NormalTok{)}\SpecialCharTok{$}\NormalTok{value\}}
\NormalTok{inverse\_cumulative }\OtherTok{\textless{}{-}} \ControlFlowTok{function}\NormalTok{(f, p, lower, upper)\{}\FunctionTok{uniroot}\NormalTok{(}\ControlFlowTok{function}\NormalTok{(x) }\FunctionTok{cumulative}\NormalTok{(f, lower, x)}\SpecialCharTok{{-}}\NormalTok{p, }\FunctionTok{c}\NormalTok{(lower, upper))}\SpecialCharTok{$}\NormalTok{root\}}

\CommentTok{\#inference functions}
\NormalTok{binom\_likelihood }\OtherTok{\textless{}{-}} \ControlFlowTok{function}\NormalTok{(prob, ...) }\FunctionTok{sapply}\NormalTok{(prob, }\ControlFlowTok{function}\NormalTok{(P)  }\FunctionTok{prod}\NormalTok{(}\FunctionTok{dbinom}\NormalTok{(}\AttributeTok{prob=}\NormalTok{P, ...)))}
\NormalTok{pois\_likelihood  }\OtherTok{\textless{}{-}} \ControlFlowTok{function}\NormalTok{(mu, ...)   }\FunctionTok{sapply}\NormalTok{(mu,   }\ControlFlowTok{function}\NormalTok{(MU) }\FunctionTok{prod}\NormalTok{(}\FunctionTok{dpois}\NormalTok{ (}\AttributeTok{lambda =}\NormalTok{ MU, ...)))}
\NormalTok{norm\_likelihood  }\OtherTok{\textless{}{-}} \ControlFlowTok{function}\NormalTok{(mu, ...)   }\FunctionTok{sapply}\NormalTok{(mu,   }\ControlFlowTok{function}\NormalTok{(MU) }\FunctionTok{prod}\NormalTok{(}\FunctionTok{dnorm}\NormalTok{ (}\AttributeTok{mean =}\NormalTok{ MU, ...) ))}

\NormalTok{posterior }\OtherTok{\textless{}{-}} \ControlFlowTok{function}\NormalTok{(parameter, prior, likelihood, lower, upper, ...) \{}
\NormalTok{  unnormalized }\OtherTok{\textless{}{-}} \ControlFlowTok{function}\NormalTok{(x) }\FunctionTok{likelihood}\NormalTok{(x, ...)}\SpecialCharTok{*}\FunctionTok{prior}\NormalTok{(x)}
\NormalTok{  norm\_factor  }\OtherTok{\textless{}{-}} \FunctionTok{integrate}\NormalTok{(unnormalized, }\AttributeTok{lower =}\NormalTok{ lower, }\AttributeTok{upper =}\NormalTok{ upper)}\SpecialCharTok{$}\NormalTok{value}
  \FunctionTok{unnormalized}\NormalTok{(parameter)}\SpecialCharTok{/}\NormalTok{norm\_factor}
\NormalTok{\}}
\end{Highlighting}
\end{Shaded}

\hypertarget{exercise-1-markov-chain-using-metropolis-hastings}{%
\section{Exercise 1: Markov Chain using
Metropolis-Hastings}\label{exercise-1-markov-chain-using-metropolis-hastings}}

\hypertarget{scenario}{%
\subsection{Scenario}\label{scenario}}

Given the following un-normalized posterior distribution
\(g(\theta | x) \propto \frac{1}{2} \exp \left( -\frac{(\theta + 3)^2}{2} \right) + \frac{1}{2} \exp \left( -\frac{(\theta - 3)^2}{2} \right)\):

\begin{itemize}
\item
  Draw a Markov Chain from the posterior distribution using a
  Metropolis-Hastings algorithm
\item
  Use a Norm (0, 1) as random-walk candidate density
\item
  Plot the sampled distribution
\item
  Analyze the chain with the CODA package and plot the chain
  autocorrelation
\item
  Try to use different burn-in cycles and thinning and plot the
  corresponding posterior distribution and the chain autocorrelation
  function. What are the best parameters ?
\end{itemize}

\hypertarget{answers}{%
\subsection{Answers}\label{answers}}

\hypertarget{algorithm}{%
\subsubsection{Algorithm}\label{algorithm}}

\begin{enumerate}
\def\labelenumi{\arabic{enumi}.}
\tightlist
\item
  Initialize the chain at some value \(\theta_0\).
\item
  Draw a random sample \(s\) from the distribution
  \(Q(s \mid \theta_t)\). This is often a multivariate Gaussian where
  \(\theta_t\) is the mean, and the covariance matrix specifies the
  typical size of steps in the chain in each dimension of the parameters
  \(\theta\).
\item
  Decide whether to accept or not the new candidate sample on the basis
  of the Metropolis ratio: \[
  \rho = \frac{f(s) Q(\theta_t \mid s)}{f(\theta_t) Q(s \mid \theta_t)}
  \]

  \begin{itemize}
  \tightlist
  \item
    If \(\rho \geq 1\), the new candidate is accepted and
    \(\theta_{t+1} = s\).
  \item
    If \(\rho < 1\), we only accept it with probability \(\rho\):

    \begin{itemize}
    \tightlist
    \item
      Draw \(u \sim U(0, 1)\) and set \(\theta_{t+1} = s\) only if
      \(u \leq \rho\).
    \item
      If \(s\) is not accepted, we set \(\theta_{t+1} = \theta_t\),
      i.e., the existing sample in the chain is repeated.
    \end{itemize}
  \end{itemize}
\end{enumerate}

\begin{Shaded}
\begin{Highlighting}[]
\CommentTok{\#{-}{-}{-}{-}{-}{-}{-}{-}{-}{-}{-}{-}{-}{-}{-}{-}{-} FUNCTION DEFINITIONS {-}{-}{-}{-}{-}{-}{-}{-}{-}{-}{-}{-}{-}{-}{-}{-}{-}{-}{-}{-}{-}{-}}
\CommentTok{\# function to generate the next point in the random walk}
\NormalTok{std }\OtherTok{\textless{}{-}} \DecValTok{1}
\NormalTok{Q\_gen }\OtherTok{\textless{}{-}} \ControlFlowTok{function}\NormalTok{(theta\_t) }\FunctionTok{rnorm}\NormalTok{(}\DecValTok{1}\NormalTok{, theta\_t, std)}
\NormalTok{Q     }\OtherTok{\textless{}{-}} \ControlFlowTok{function}\NormalTok{(x, theta\_t) }\FunctionTok{dnorm}\NormalTok{(x, theta\_t, std)}

\CommentTok{\# unnormalized posterior}
\NormalTok{g }\OtherTok{\textless{}{-}} \ControlFlowTok{function}\NormalTok{(theta) }\FloatTok{0.5} \SpecialCharTok{*} \FunctionTok{exp}\NormalTok{(}\SpecialCharTok{{-}}\FloatTok{0.5} \SpecialCharTok{*}\NormalTok{ (theta }\SpecialCharTok{+} \DecValTok{3}\NormalTok{)}\SpecialCharTok{\^{}}\DecValTok{2}\NormalTok{) }\SpecialCharTok{+} \FloatTok{0.5} \SpecialCharTok{*} \FunctionTok{exp}\NormalTok{(}\SpecialCharTok{{-}}\FloatTok{0.5} \SpecialCharTok{*}\NormalTok{ (theta }\SpecialCharTok{{-}} \DecValTok{3}\NormalTok{)}\SpecialCharTok{\^{}}\DecValTok{2}\NormalTok{)}

\NormalTok{generate }\OtherTok{\textless{}{-}} 
    \ControlFlowTok{function}\NormalTok{(theta\_t)\{}
\NormalTok{      s }\OtherTok{\textless{}{-}} \FunctionTok{Q\_gen}\NormalTok{(theta\_t)}
\NormalTok{      rho }\OtherTok{\textless{}{-}} \FunctionTok{g}\NormalTok{(s)}\SpecialCharTok{*}\FunctionTok{Q}\NormalTok{(theta\_t, s)}\SpecialCharTok{/}\NormalTok{(}\FunctionTok{g}\NormalTok{(theta\_t)}\SpecialCharTok{*}\FunctionTok{Q}\NormalTok{(s, theta\_t))}
      \FunctionTok{ifelse}\NormalTok{ (rho}\SpecialCharTok{\textless{}}\DecValTok{1}\NormalTok{, \{ u }\OtherTok{\textless{}{-}} \FunctionTok{runif}\NormalTok{(}\DecValTok{1}\NormalTok{); }\FunctionTok{ifelse}\NormalTok{(u }\SpecialCharTok{\textless{}=}\NormalTok{ rho, s, theta\_t)\}, s)}
\NormalTok{    \}}

\NormalTok{Metropolis\_Hastings }\OtherTok{\textless{}{-}} \ControlFlowTok{function}\NormalTok{(N, theta\_0, }\AttributeTok{burnin =} \DecValTok{0}\NormalTok{, }\AttributeTok{thinning =} \DecValTok{1}\NormalTok{)\{}
\NormalTok{  chain }\OtherTok{\textless{}{-}} \FunctionTok{vector}\NormalTok{(}\AttributeTok{length =}\NormalTok{ N)}
\NormalTok{  theta\_t }\OtherTok{\textless{}{-}}\NormalTok{ theta\_0}
  
  \CommentTok{\# burn{-}in phase}
  \ControlFlowTok{for}\NormalTok{ (ib }\ControlFlowTok{in} \DecValTok{0}\SpecialCharTok{:}\NormalTok{burnin) \{}
\NormalTok{      theta\_t }\OtherTok{\textless{}{-}} \FunctionTok{generate}\NormalTok{(theta\_t)}
\NormalTok{  \}}
  
  \CommentTok{\# save phase}
  \ControlFlowTok{for}\NormalTok{(step }\ControlFlowTok{in} \DecValTok{1}\SpecialCharTok{:}\NormalTok{N)\{}
    \CommentTok{\#thinning}
    \ControlFlowTok{for}\NormalTok{(it }\ControlFlowTok{in} \DecValTok{1}\SpecialCharTok{:}\NormalTok{thinning)\{}
\NormalTok{      theta\_t }\OtherTok{\textless{}{-}} \FunctionTok{generate}\NormalTok{(theta\_t)}
\NormalTok{    \}}
    \CommentTok{\#append in the chain}
\NormalTok{    chain[step] }\OtherTok{\textless{}{-}}\NormalTok{ theta\_t}
\NormalTok{  \}}
  
  \FunctionTok{return}\NormalTok{(chain)}
\NormalTok{\}}
\end{Highlighting}
\end{Shaded}

\begin{itemize}
\item
  Draw a Markov Chain from the posterior distribution using a
  Metropolis-Hastings algorithm
\item
  Use a Norm (0, 1) as random-walk candidate density
\item
  Plot the sampled distribution
\end{itemize}

\begin{Shaded}
\begin{Highlighting}[]
\NormalTok{N }\OtherTok{\textless{}{-}} \DecValTok{100000}
\NormalTok{theta\_0 }\OtherTok{\textless{}{-}} \DecValTok{0}

\NormalTok{chain }\OtherTok{\textless{}{-}} \FunctionTok{Metropolis\_Hastings}\NormalTok{(N, theta\_0)}

\NormalTok{df }\OtherTok{\textless{}{-}} \FunctionTok{data.frame}\NormalTok{(}\AttributeTok{chain =}\NormalTok{ chain)}

\NormalTok{plt }\OtherTok{\textless{}{-}} \FunctionTok{ggplot}\NormalTok{(df, }\FunctionTok{aes}\NormalTok{(}\AttributeTok{x =}\NormalTok{ chain, }\AttributeTok{y=}\FunctionTok{after\_stat}\NormalTok{(density))) }\SpecialCharTok{+}
  \FunctionTok{geom\_histogram}\NormalTok{(}\AttributeTok{bins =} \DecValTok{50}\NormalTok{, }\AttributeTok{fill=}\NormalTok{color\_vector[}\DecValTok{5}\NormalTok{], }\AttributeTok{alpha =} \FloatTok{1.}\NormalTok{, }\AttributeTok{color=}\StringTok{\textquotesingle{}black\textquotesingle{}}\NormalTok{) }\SpecialCharTok{+}
  \FunctionTok{geom\_density}\NormalTok{(}\AttributeTok{color =}\NormalTok{ color\_vector[}\DecValTok{7}\NormalTok{], }\AttributeTok{size =} \DecValTok{1}\NormalTok{) }\SpecialCharTok{+}
  \FunctionTok{labs}\NormalTok{(}\AttributeTok{title =} \StringTok{"Histogram and Density of Metropolis{-}Hastings Chain"}\NormalTok{, }\AttributeTok{x =} \FunctionTok{expression}\NormalTok{(theta), }\AttributeTok{y =} \StringTok{"Posterior"}\NormalTok{)}
\end{Highlighting}
\end{Shaded}

\begin{verbatim}
## Warning: Using `size` aesthetic for lines was deprecated in ggplot2 3.4.0.
## i Please use `linewidth` instead.
## This warning is displayed once every 8 hours.
## Call `lifecycle::last_lifecycle_warnings()` to see where this warning was
## generated.
\end{verbatim}

\begin{Shaded}
\begin{Highlighting}[]
\CommentTok{\# Display the plot}
\FunctionTok{print}\NormalTok{(plt)}
\end{Highlighting}
\end{Shaded}

\includegraphics{Emanuele_Coradin_Rlab05_files/figure-latex/1.1 - 1.3-1.pdf}

\begin{itemize}
\item
  Analyze the chain with the CODA package and plot the chain
  autocorrelation
\item
  Try to use different burn-in cycles and thinning and plot the
  corresponding posterior distribution and the chain autocorrelation
  function. What are the best parameters ?
\end{itemize}

\begin{Shaded}
\begin{Highlighting}[]
\CommentTok{\# Convert the chain to an mcmc object}
\NormalTok{mcmc\_chain }\OtherTok{\textless{}{-}} \FunctionTok{mcmc}\NormalTok{(chain)}

\NormalTok{lags }\OtherTok{\textless{}{-}} \FunctionTok{seq}\NormalTok{(}\DecValTok{0}\NormalTok{, }\DecValTok{500}\NormalTok{, }\DecValTok{10}\NormalTok{)}
\NormalTok{autocorr\_chain }\OtherTok{\textless{}{-}} \FunctionTok{autocorr}\NormalTok{(mcmc\_chain, }\AttributeTok{lags =}\NormalTok{ lags)}

\FunctionTok{plot}\NormalTok{(lags, autocorr\_chain, }\AttributeTok{type =} \StringTok{\textquotesingle{}s\textquotesingle{}}\NormalTok{, }\AttributeTok{col=}\NormalTok{color\_vector[}\DecValTok{7}\NormalTok{], }\AttributeTok{lty=}\DecValTok{1}\NormalTok{, }\AttributeTok{lwd=}\DecValTok{2}\NormalTok{, }\AttributeTok{main =} \StringTok{\textquotesingle{}Autocorrelation of the chain\textquotesingle{}}\NormalTok{, }\AttributeTok{xlab =} \StringTok{\textquotesingle{}lags\textquotesingle{}}\NormalTok{, }\AttributeTok{ylab=}\StringTok{\textquotesingle{}autocorrelation\textquotesingle{}}\NormalTok{)}
\end{Highlighting}
\end{Shaded}

\includegraphics{Emanuele_Coradin_Rlab05_files/figure-latex/1.4-1.pdf}

\begin{verbatim}

#{r 1.5}

burnin_list   <- list(100, 1000, 10000)
thinning_list <- list(5, 10, 50)

chain_list <- unlist(lapply(burnin_list, function(burnin) lapply(thinning_list, function(thinning) Metropolis_Hastings(N, theta_0, burnin, thinning))), recursive=FALSE)

# Save the chain_list to a file
SaveRDS(chain_list, 'chain_list_RDS.RData')
\end{verbatim}

\begin{Shaded}
\begin{Highlighting}[]
\NormalTok{burnin\_list   }\OtherTok{\textless{}{-}} \FunctionTok{list}\NormalTok{(}\DecValTok{100}\NormalTok{, }\DecValTok{1000}\NormalTok{, }\DecValTok{10000}\NormalTok{)}
\NormalTok{thinning\_list }\OtherTok{\textless{}{-}} \FunctionTok{list}\NormalTok{(}\DecValTok{5}\NormalTok{, }\DecValTok{10}\NormalTok{, }\DecValTok{50}\NormalTok{)}

\CommentTok{\# Retrieve the precomputed chains}
\NormalTok{chain\_list\_RDS }\OtherTok{\textless{}{-}} \FunctionTok{readRDS}\NormalTok{(}\StringTok{"chain\_list\_RDS.RData"}\NormalTok{)}

\CommentTok{\# Plot the autocorrelation}

\NormalTok{mcmc\_chain\_list }\OtherTok{\textless{}{-}} \FunctionTok{lapply}\NormalTok{(chain\_list\_RDS,  mcmc)}
\NormalTok{N\_chains }\OtherTok{\textless{}{-}} \FunctionTok{length}\NormalTok{(mcmc\_chain\_list)}
  
\NormalTok{lags }\OtherTok{\textless{}{-}} \FunctionTok{seq}\NormalTok{(}\DecValTok{0}\NormalTok{, }\DecValTok{150}\NormalTok{, }\DecValTok{5}\NormalTok{)}
\NormalTok{autocorr\_chain\_list }\OtherTok{\textless{}{-}} \FunctionTok{lapply}\NormalTok{(mcmc\_chain\_list, }\ControlFlowTok{function}\NormalTok{(mcmc\_chain) }\FunctionTok{log}\NormalTok{(}\FunctionTok{abs}\NormalTok{(}\FunctionTok{autocorr}\NormalTok{(mcmc\_chain, }\AttributeTok{lags =}\NormalTok{ lags))))}

\NormalTok{ylim}\OtherTok{=}\FunctionTok{c}\NormalTok{(}\FunctionTok{min}\NormalTok{(}\FunctionTok{unlist}\NormalTok{(autocorr\_chain\_list)), }\FunctionTok{max}\NormalTok{(}\FunctionTok{unlist}\NormalTok{(autocorr\_chain\_list))}\SpecialCharTok{*}\FloatTok{1.1}\NormalTok{)}

\NormalTok{vanilla\_mcmc }\OtherTok{\textless{}{-}} \FunctionTok{log}\NormalTok{(}\FunctionTok{abs}\NormalTok{(}\FunctionTok{autocorr}\NormalTok{(mcmc\_chain, }\AttributeTok{lags =}\NormalTok{ lags)))}

\NormalTok{colormap }\OtherTok{\textless{}{-}} \FunctionTok{rainbow}\NormalTok{(N\_chains)}

\NormalTok{labels }\OtherTok{\textless{}{-}} \FunctionTok{unlist}\NormalTok{(}\FunctionTok{lapply}\NormalTok{(burnin\_list, }\ControlFlowTok{function}\NormalTok{(burnin) }
                 \FunctionTok{lapply}\NormalTok{(thinning\_list, }\ControlFlowTok{function}\NormalTok{(thinning) }
                 \FunctionTok{paste}\NormalTok{(}\StringTok{"B ="}\NormalTok{, burnin, }\StringTok{", T ="}\NormalTok{, thinning))))}

\FunctionTok{plot}\NormalTok{(lags, vanilla\_mcmc, }\AttributeTok{type =} \StringTok{\textquotesingle{}s\textquotesingle{}}\NormalTok{, }\AttributeTok{col=}\NormalTok{color\_vector[}\DecValTok{7}\NormalTok{], }\AttributeTok{lty=}\DecValTok{1}\NormalTok{, }\AttributeTok{lwd=}\DecValTok{2}\NormalTok{, }\AttributeTok{main =} \StringTok{\textquotesingle{}Autocorrelation of the chains\textquotesingle{}}\NormalTok{, }\AttributeTok{xlab =} \StringTok{\textquotesingle{}lags\textquotesingle{}}\NormalTok{, }\AttributeTok{ylab=}\StringTok{\textquotesingle{}log(|autocorrelation|)\textquotesingle{}}\NormalTok{, }\AttributeTok{xlim =} \FunctionTok{c}\NormalTok{(}\DecValTok{0}\NormalTok{,}\DecValTok{205}\NormalTok{), }\AttributeTok{ylim =}\NormalTok{ ylim)}

\NormalTok{void }\OtherTok{\textless{}{-}} \FunctionTok{sapply}\NormalTok{(}\DecValTok{1}\SpecialCharTok{:}\NormalTok{N\_chains, }\ControlFlowTok{function}\NormalTok{(iline) }\FunctionTok{lines}\NormalTok{(lags, autocorr\_chain\_list[[iline]], }\AttributeTok{type =} \StringTok{\textquotesingle{}s\textquotesingle{}}\NormalTok{, }\AttributeTok{col=}\NormalTok{colormap[iline], }\AttributeTok{lty=}\NormalTok{iline, }\AttributeTok{lwd=}\DecValTok{2}\NormalTok{))}

\CommentTok{\# Place the legend outside the plot area, using multiple columns}
\FunctionTok{legend}\NormalTok{(}\StringTok{\textquotesingle{}topright\textquotesingle{}}\NormalTok{, }\AttributeTok{legend =}\NormalTok{ labels, }\AttributeTok{col =}\NormalTok{ colormap, }\AttributeTok{lty =} \DecValTok{1}\SpecialCharTok{:}\NormalTok{N\_chains, }\AttributeTok{lwd =} \DecValTok{2}\NormalTok{, }\AttributeTok{ncol =} \DecValTok{1}\NormalTok{, }\AttributeTok{cex =} \FloatTok{0.8}\NormalTok{)}
\end{Highlighting}
\end{Shaded}

\includegraphics{Emanuele_Coradin_Rlab05_files/figure-latex/1.5 plots-1.pdf}

\begin{Shaded}
\begin{Highlighting}[]
\CommentTok{\# Plot the posterior}
\NormalTok{plot\_posterior }\OtherTok{\textless{}{-}} \ControlFlowTok{function}\NormalTok{(chain, color)\{}
  \FunctionTok{hist}\NormalTok{(chain, }\AttributeTok{breaks =} \FunctionTok{c}\NormalTok{(}\SpecialCharTok{{-}}\ConstantTok{Inf}\NormalTok{, posterior\_breaks, }\ConstantTok{Inf}\NormalTok{), }\AttributeTok{freq =} \ConstantTok{FALSE}\NormalTok{, }\AttributeTok{col =} \ConstantTok{NULL}\NormalTok{, }\AttributeTok{border =}\NormalTok{ color, }\AttributeTok{add=}\ConstantTok{TRUE}\NormalTok{, }\AttributeTok{xlab =} \FunctionTok{expression}\NormalTok{(theta), }\AttributeTok{ylab =} \StringTok{\textquotesingle{}density\textquotesingle{}}\NormalTok{, }\AttributeTok{main =} \StringTok{\textquotesingle{}Histogram of the posteriors from different chains\textquotesingle{}}\NormalTok{)}
\NormalTok{\}}

\NormalTok{posterior\_breaks }\OtherTok{\textless{}{-}} \FunctionTok{hist}\NormalTok{(}\AttributeTok{x=}\NormalTok{chain, }\AttributeTok{breaks =} \DecValTok{50}\NormalTok{, }\AttributeTok{freq =} \ConstantTok{FALSE}\NormalTok{, }\AttributeTok{col =} \ConstantTok{NULL}\NormalTok{, }\AttributeTok{border =}\NormalTok{ color\_vector[}\DecValTok{7}\NormalTok{], }\AttributeTok{xlim =} \FunctionTok{c}\NormalTok{(}\SpecialCharTok{{-}}\DecValTok{7}\NormalTok{, }\DecValTok{10}\NormalTok{))}\SpecialCharTok{$}\NormalTok{breaks}

\NormalTok{void }\OtherTok{\textless{}{-}} \FunctionTok{mapply}\NormalTok{(plot\_posterior, chain\_list\_RDS, colormap)}
\FunctionTok{legend}\NormalTok{(}\StringTok{\textquotesingle{}topright\textquotesingle{}}\NormalTok{, }\AttributeTok{legend =}\NormalTok{ labels, }\AttributeTok{col =}\NormalTok{ colormap, }\AttributeTok{lty =} \DecValTok{1}\NormalTok{, }\AttributeTok{lwd =} \DecValTok{2}\NormalTok{, }\AttributeTok{ncol =} \DecValTok{1}\NormalTok{, }\AttributeTok{cex =} \FloatTok{0.8}\NormalTok{)}
\end{Highlighting}
\end{Shaded}

\includegraphics{Emanuele_Coradin_Rlab05_files/figure-latex/1.5 plots-2.pdf}

From this plot we can see that it seems that every chain reaches
convergence, no matter the different level of autocorrelation.

\hypertarget{exercise-2-mcmc-inference-on-linear-model}{%
\section{Exercise 2: MCMC inference on Linear
model}\label{exercise-2-mcmc-inference-on-linear-model}}

\hypertarget{scenario-1}{%
\subsection{Scenario}\label{scenario-1}}

A set of measured data should follow, according to the physics model
applied to them, a linear behavior. Data are the following: Y = \{
-7.821 -1.494 -15.444 -10.807 -13.735 -14.442 -15.892 -18.326 \} X = \{
5 6 7 8 9 10 11 12 \}

Tasks:

\begin{itemize}
\item
  Perform a simple linear regression model running a Markov Chain Monte
  Carlo with JAGS, assuming that data follow the model: Z{[}i{]} = a + b
  * X{[}i{]}; and the likelihood of the measured data follow a Gaussian
  likelihood distribution: Y{[}i{]} dnorm(Z{[}i{]}, c) (you can
  constrain the parameter a, b and c to the following intervals: a ∈
  {[}1, 10{]}, b ∈ {[}−1, 3{]} and c ∈ {[}0.034, 4{]})
\item
  Run JAGS experimenting with the burn-in and number of iterations of
  the chain. Plot the evolution of the chains and the posterior
  distributions of a and b. Compute the 95\% credibility interval for
  the parameters.
\item
  Using the obtained posterior distributions, compute the posterior
  distribution of \(\sigma = \sqrt{(\frac{1}{c})}\).
\end{itemize}

\hypertarget{answers-1}{%
\subsection{Answers}\label{answers-1}}

\begin{Shaded}
\begin{Highlighting}[]
\NormalTok{plot\_intervals }\OtherTok{\textless{}{-}} \ControlFlowTok{function}\NormalTok{(histogram)\{}
  \CommentTok{\# Compute:}
\NormalTok{  xlim  }\OtherTok{=} \FunctionTok{c}\NormalTok{(}\FunctionTok{min}\NormalTok{(histogram}\SpecialCharTok{$}\NormalTok{breaks), }\FunctionTok{max}\NormalTok{(histogram}\SpecialCharTok{$}\NormalTok{breaks))}
\NormalTok{  step\_posterior }\OtherTok{\textless{}{-}} \FunctionTok{stepfun}\NormalTok{(histogram}\SpecialCharTok{$}\NormalTok{breaks, }\FunctionTok{c}\NormalTok{(}\DecValTok{0}\NormalTok{, histogram}\SpecialCharTok{$}\NormalTok{density, }\DecValTok{0}\NormalTok{))}
\NormalTok{  mean\_posterior }\OtherTok{\textless{}{-}} \FunctionTok{mean\_pdf}\NormalTok{(step\_posterior, }\AttributeTok{lower =}\NormalTok{ xlim[}\DecValTok{1}\NormalTok{], }\AttributeTok{upper =}\NormalTok{ xlim[}\DecValTok{2}\NormalTok{])}
\NormalTok{  std\_posterior  }\OtherTok{\textless{}{-}} \FunctionTok{std\_pdf}\NormalTok{ (step\_posterior, }\AttributeTok{lower =}\NormalTok{ xlim[}\DecValTok{1}\NormalTok{], }\AttributeTok{upper =}\NormalTok{ xlim[}\DecValTok{2}\NormalTok{])}
\NormalTok{  interval\_95    }\OtherTok{\textless{}{-}} \FunctionTok{sapply}\NormalTok{(}\FunctionTok{c}\NormalTok{(}\FloatTok{0.025}\NormalTok{, }\FloatTok{0.975}\NormalTok{), }\ControlFlowTok{function}\NormalTok{(P) }\FunctionTok{inverse\_cumulative}\NormalTok{(step\_posterior, }\AttributeTok{p =}\NormalTok{ P, }\AttributeTok{lower =}\NormalTok{ xlim[}\DecValTok{1}\NormalTok{], }\AttributeTok{upper =}\NormalTok{ xlim[}\DecValTok{2}\NormalTok{]))}
  
  \CommentTok{\# Plot:}
\NormalTok{  x\_plot }\OtherTok{\textless{}{-}} \FunctionTok{seq}\NormalTok{(}\AttributeTok{from=}\NormalTok{interval\_95[}\DecValTok{1}\NormalTok{], }\AttributeTok{to=}\NormalTok{interval\_95[}\DecValTok{2}\NormalTok{], }\AttributeTok{length.out=}\DecValTok{500}\NormalTok{)}
\NormalTok{  y\_plot }\OtherTok{\textless{}{-}} \FunctionTok{c}\NormalTok{(}\DecValTok{0}\NormalTok{, }\FunctionTok{step\_posterior}\NormalTok{(x\_plot), }\DecValTok{0}\NormalTok{)}
\NormalTok{  x\_plot }\OtherTok{\textless{}{-}} \FunctionTok{c}\NormalTok{(interval\_95[}\DecValTok{1}\NormalTok{], x\_plot, interval\_95[}\DecValTok{2}\NormalTok{])}
  \FunctionTok{polygon}\NormalTok{(x\_plot, y\_plot, }\AttributeTok{col =} \FunctionTok{adjustcolor}\NormalTok{(color\_vector[}\DecValTok{7}\NormalTok{], }\AttributeTok{alpha.f =} \FloatTok{0.25}\NormalTok{),}\AttributeTok{border =} \ConstantTok{NA}\NormalTok{)}
  
  \FunctionTok{abline}\NormalTok{(}\AttributeTok{v=}\NormalTok{mean\_posterior, }\AttributeTok{col =}\NormalTok{ color\_vector[}\DecValTok{1}\NormalTok{], }\AttributeTok{lwd=}\DecValTok{2}\NormalTok{, }\AttributeTok{lty=}\StringTok{\textquotesingle{}longdash\textquotesingle{}}\NormalTok{)}
  \FunctionTok{abline}\NormalTok{(}\AttributeTok{v=}\NormalTok{mean\_posterior }\SpecialCharTok{{-}}\NormalTok{ std\_posterior, }\AttributeTok{col =}\NormalTok{ color\_vector[}\DecValTok{6}\NormalTok{], }\AttributeTok{lwd=}\DecValTok{2}\NormalTok{, }\AttributeTok{lty=}\StringTok{\textquotesingle{}dashed\textquotesingle{}}\NormalTok{)}
  \FunctionTok{abline}\NormalTok{(}\AttributeTok{v=}\NormalTok{mean\_posterior }\SpecialCharTok{+}\NormalTok{ std\_posterior, }\AttributeTok{col =}\NormalTok{ color\_vector[}\DecValTok{6}\NormalTok{], }\AttributeTok{lwd=}\DecValTok{2}\NormalTok{, }\AttributeTok{lty=}\StringTok{\textquotesingle{}dashed\textquotesingle{}}\NormalTok{)}
  
  \FunctionTok{legend}\NormalTok{(}\StringTok{"topright"}\NormalTok{, }\AttributeTok{legend =} \FunctionTok{c}\NormalTok{(}\StringTok{"Mean"}\NormalTok{, }\StringTok{"±std"}\NormalTok{), }\AttributeTok{col =} \FunctionTok{c}\NormalTok{(color\_vector[}\DecValTok{1}\NormalTok{], color\_vector[}\DecValTok{6}\NormalTok{]), }\AttributeTok{lty =} \FunctionTok{c}\NormalTok{(}\StringTok{\textquotesingle{}longdash\textquotesingle{}}\NormalTok{, }\StringTok{\textquotesingle{}dashed\textquotesingle{}}\NormalTok{))}
  
  \FunctionTok{legend}\NormalTok{(}\StringTok{"right"}\NormalTok{, }\AttributeTok{legend=}\StringTok{"95\%\% credibility interval"}\NormalTok{, }\AttributeTok{fill=}\NormalTok{color\_vector[}\DecValTok{7}\NormalTok{])}
  
  \FunctionTok{return}\NormalTok{()}
  
\NormalTok{\}}
\end{Highlighting}
\end{Shaded}

\begin{Shaded}
\begin{Highlighting}[]
\CommentTok{\# Defining data}
\NormalTok{X }\OtherTok{\textless{}{-}} \FunctionTok{c}\NormalTok{(}\DecValTok{5}\NormalTok{, }\DecValTok{6}\NormalTok{, }\DecValTok{7}\NormalTok{, }\DecValTok{8}\NormalTok{, }\DecValTok{9}\NormalTok{, }\DecValTok{10}\NormalTok{, }\DecValTok{11}\NormalTok{, }\DecValTok{12}\NormalTok{)}
\NormalTok{Y }\OtherTok{\textless{}{-}} \FunctionTok{c}\NormalTok{(}\SpecialCharTok{{-}}\FloatTok{7.821}\NormalTok{, }\SpecialCharTok{{-}}\FloatTok{1.494}\NormalTok{, }\SpecialCharTok{{-}}\FloatTok{15.444}\NormalTok{, }\SpecialCharTok{{-}}\FloatTok{10.807}\NormalTok{, }\SpecialCharTok{{-}}\FloatTok{13.735}\NormalTok{, }\SpecialCharTok{{-}}\FloatTok{14.442}\NormalTok{, }\SpecialCharTok{{-}}\FloatTok{15.892}\NormalTok{, }\SpecialCharTok{{-}}\FloatTok{18.326}\NormalTok{) }
\NormalTok{N\_burnin }\OtherTok{\textless{}{-}} \DecValTok{2000} \CommentTok{\# length of the burn{-}in phase }
\NormalTok{thinning }\OtherTok{\textless{}{-}} \DecValTok{7}
\NormalTok{Nrep }\OtherTok{=} \DecValTok{10000}     \CommentTok{\# number of values to simulate}

\CommentTok{\# Let\textquotesingle{} define the model}
\NormalTok{model\_string }\OtherTok{\textless{}{-}} \StringTok{"model\{}
\StringTok{  }
\StringTok{  \# Model: Z[i] = a + b * X[i];}
\StringTok{  \# Likelihood: Y[i] \textasciitilde{}dnorm(Z[i], c)}
\StringTok{  \# a ∈ [1, 10], b ∈ [−1, 3] and c ∈ [0.034, 4]) }

\StringTok{  \# Likelihood}
\StringTok{  for (i in 1:length(X)) \{}
\StringTok{    Z[i] \textless{}{-} a + b * X[i]}
\StringTok{    Y[i] \textasciitilde{} dnorm(Z[i], c)}
\StringTok{  \}}

\StringTok{  \# Prior}
\StringTok{  a \textasciitilde{} dunif({-}3, 10)}
\StringTok{  b \textasciitilde{} dunif({-}3, 3)}
\StringTok{  c \textasciitilde{} dunif(0.034, 4)}
\StringTok{  }
\StringTok{\}"}

\CommentTok{\# Compile jags model}
\NormalTok{dataList }\OtherTok{=} \FunctionTok{list}\NormalTok{(}\AttributeTok{X =}\NormalTok{ X, }\AttributeTok{Y =}\NormalTok{ Y)}
\NormalTok{model }\OtherTok{\textless{}{-}} \FunctionTok{jags.model}\NormalTok{(}\AttributeTok{file =} \FunctionTok{textConnection}\NormalTok{(model\_string), }
                    \AttributeTok{data =}\NormalTok{ dataList)}
\end{Highlighting}
\end{Shaded}

\begin{verbatim}
## Compiling model graph
##    Resolving undeclared variables
##    Allocating nodes
## Graph information:
##    Observed stochastic nodes: 8
##    Unobserved stochastic nodes: 3
##    Total graph size: 40
## 
## Initializing model
\end{verbatim}

\begin{Shaded}
\begin{Highlighting}[]
\CommentTok{\# Add burnin}
\FunctionTok{update}\NormalTok{(model, }\AttributeTok{n.iter =}\NormalTok{ N\_burnin)}

\CommentTok{\# Sample the posterior}
\NormalTok{posterior\_sample }\OtherTok{\textless{}{-}} \FunctionTok{coda.samples}\NormalTok{(model,}
                       \AttributeTok{variable.names =} \FunctionTok{c}\NormalTok{(}\StringTok{"a"}\NormalTok{, }\StringTok{"b"}\NormalTok{, }\StringTok{"c"}\NormalTok{),}
                       \AttributeTok{n.iter =}\NormalTok{ Nrep, }\AttributeTok{thin =}\NormalTok{ thinning)}
\FunctionTok{summary}\NormalTok{(posterior\_sample)}
\end{Highlighting}
\end{Shaded}

\begin{verbatim}
## 
## Iterations = 3007:12996
## Thinning interval = 7 
## Number of chains = 1 
## Sample size per chain = 1428 
## 
## 1. Empirical mean and standard deviation for each variable,
##    plus standard error of the mean:
## 
##      Mean      SD Naive SE Time-series SE
## a  3.0795 3.15584 0.083512       0.163585
## b -1.8012 0.36290 0.009603       0.018094
## c  0.1104 0.05104 0.001351       0.001283
## 
## 2. Quantiles for each variable:
## 
##      2.5%      25%      50%     75%   97.5%
## a -2.4440  0.68398  2.94979  5.4451  9.1601
## b -2.5135 -2.05512 -1.79559 -1.5276 -1.1415
## c  0.0406  0.07242  0.09966  0.1369  0.2315
\end{verbatim}

\begin{Shaded}
\begin{Highlighting}[]
\FunctionTok{plot}\NormalTok{(posterior\_sample)}
\end{Highlighting}
\end{Shaded}

\includegraphics{Emanuele_Coradin_Rlab05_files/figure-latex/2-1.pdf}

\begin{Shaded}
\begin{Highlighting}[]
\NormalTok{posterior\_matrix }\OtherTok{\textless{}{-}} \FunctionTok{as.matrix}\NormalTok{(posterior\_sample)}
\CommentTok{\# Retrieve the chains}
\NormalTok{a\_samples }\OtherTok{\textless{}{-}}\NormalTok{ posterior\_matrix[, }\StringTok{"a"}\NormalTok{]}
\NormalTok{b\_samples }\OtherTok{\textless{}{-}}\NormalTok{ posterior\_matrix[, }\StringTok{"b"}\NormalTok{]}
\NormalTok{c\_samples }\OtherTok{\textless{}{-}}\NormalTok{ posterior\_matrix[, }\StringTok{"c"}\NormalTok{]}

\CommentTok{\# Set up the plotting area to have 2 rows and 2 column}
\FunctionTok{par}\NormalTok{(}\AttributeTok{mfrow =} \FunctionTok{c}\NormalTok{(}\DecValTok{2}\NormalTok{, }\DecValTok{2}\NormalTok{))  }

\FunctionTok{acf}\NormalTok{(a\_samples, }\AttributeTok{main =} \StringTok{"Autocorrelation of a"}\NormalTok{)}
\FunctionTok{acf}\NormalTok{(b\_samples, }\AttributeTok{main =} \StringTok{"Autocorrelation of b"}\NormalTok{)}
\FunctionTok{acf}\NormalTok{(c\_samples, }\AttributeTok{main =} \StringTok{"Autocorrelation of c"}\NormalTok{)}
\end{Highlighting}
\end{Shaded}

\includegraphics{Emanuele_Coradin_Rlab05_files/figure-latex/2-2.pdf}

\begin{Shaded}
\begin{Highlighting}[]
\CommentTok{\# Computing the histograms and the step functions}

\NormalTok{a\_hist }\OtherTok{\textless{}{-}} \FunctionTok{hist}\NormalTok{(a\_samples, }\AttributeTok{breaks =} \DecValTok{30}\NormalTok{, }\AttributeTok{main =} \StringTok{"Posterior Distribution of a"}\NormalTok{, }\AttributeTok{xlab =} \StringTok{"a"}\NormalTok{, }\AttributeTok{freq =} \ConstantTok{FALSE}\NormalTok{, }\AttributeTok{col =} \ConstantTok{NULL}\NormalTok{, }\AttributeTok{border =}\NormalTok{ color\_vector[}\DecValTok{6}\NormalTok{])}
\NormalTok{a\_best }\OtherTok{\textless{}{-}}\NormalTok{ a\_hist}\SpecialCharTok{$}\NormalTok{mids[ }\FunctionTok{which.max}\NormalTok{(a\_hist}\SpecialCharTok{$}\NormalTok{density) ]}
\FunctionTok{plot\_intervals}\NormalTok{(a\_hist)}
\end{Highlighting}
\end{Shaded}

\includegraphics{Emanuele_Coradin_Rlab05_files/figure-latex/2 hist-1.pdf}

\begin{verbatim}
## NULL
\end{verbatim}

\begin{Shaded}
\begin{Highlighting}[]
\NormalTok{b\_hist }\OtherTok{\textless{}{-}} \FunctionTok{hist}\NormalTok{(b\_samples, }\AttributeTok{breaks =} \DecValTok{30}\NormalTok{, }\AttributeTok{main =} \StringTok{"Posterior Distribution of b"}\NormalTok{, }\AttributeTok{xlab =} \StringTok{"b"}\NormalTok{, }\AttributeTok{freq =} \ConstantTok{FALSE}\NormalTok{, }\AttributeTok{col =} \ConstantTok{NULL}\NormalTok{, }\AttributeTok{border =}\NormalTok{ color\_vector[}\DecValTok{6}\NormalTok{], }\AttributeTok{xlim=}\FunctionTok{c}\NormalTok{(}\SpecialCharTok{{-}}\DecValTok{3}\NormalTok{, }\DecValTok{0}\NormalTok{))}
\NormalTok{b\_best }\OtherTok{\textless{}{-}}\NormalTok{ b\_hist}\SpecialCharTok{$}\NormalTok{mids[ }\FunctionTok{which.max}\NormalTok{(b\_hist}\SpecialCharTok{$}\NormalTok{density) ]}
\FunctionTok{plot\_intervals}\NormalTok{(b\_hist)}
\end{Highlighting}
\end{Shaded}

\includegraphics{Emanuele_Coradin_Rlab05_files/figure-latex/2 hist-2.pdf}

\begin{verbatim}
## NULL
\end{verbatim}

\begin{Shaded}
\begin{Highlighting}[]
\NormalTok{c\_hist }\OtherTok{\textless{}{-}} \FunctionTok{hist}\NormalTok{(c\_samples, }\AttributeTok{breaks =} \DecValTok{30}\NormalTok{, }\AttributeTok{main =} \StringTok{"Posterior Distribution of c"}\NormalTok{, }\AttributeTok{xlab =} \StringTok{"c"}\NormalTok{, }\AttributeTok{freq =} \ConstantTok{FALSE}\NormalTok{, }\AttributeTok{col =} \ConstantTok{NULL}\NormalTok{, }\AttributeTok{border =}\NormalTok{ color\_vector[}\DecValTok{6}\NormalTok{])}
\NormalTok{c\_best }\OtherTok{\textless{}{-}}\NormalTok{ c\_hist}\SpecialCharTok{$}\NormalTok{mids[ }\FunctionTok{which.max}\NormalTok{(c\_hist}\SpecialCharTok{$}\NormalTok{density) ]}
\FunctionTok{plot\_intervals}\NormalTok{(c\_hist)}
\end{Highlighting}
\end{Shaded}

\includegraphics{Emanuele_Coradin_Rlab05_files/figure-latex/2 hist-3.pdf}

\begin{verbatim}
## NULL
\end{verbatim}

\begin{Shaded}
\begin{Highlighting}[]
\CommentTok{\# {-}{-}{-}{-}{-} Linear Fit plot {-}{-}{-}{-}{-}}

\FunctionTok{plot}\NormalTok{(X, Y, }\AttributeTok{main=}\StringTok{"Linear fit with Bayesian inference"}\NormalTok{, }\AttributeTok{col=}\NormalTok{color\_vector[}\DecValTok{6}\NormalTok{], }\AttributeTok{pch=}\DecValTok{19}\NormalTok{)}
\FunctionTok{curve}\NormalTok{(b\_best}\SpecialCharTok{*}\NormalTok{x}\SpecialCharTok{+}\NormalTok{a\_best, }\AttributeTok{from =} \DecValTok{0}\NormalTok{, }\AttributeTok{to =} \DecValTok{13}\NormalTok{, }\AttributeTok{add=}\ConstantTok{TRUE}\NormalTok{, }\AttributeTok{col=}\NormalTok{color\_vector[}\DecValTok{1}\NormalTok{], }\AttributeTok{lwd=}\DecValTok{2}\NormalTok{)}
\end{Highlighting}
\end{Shaded}

\includegraphics{Emanuele_Coradin_Rlab05_files/figure-latex/2 hist-4.pdf}

\begin{Shaded}
\begin{Highlighting}[]
\NormalTok{sigma\_samples }\OtherTok{\textless{}{-}} \FunctionTok{sqrt}\NormalTok{(}\DecValTok{1}\SpecialCharTok{/}\NormalTok{c\_samples)}

\FunctionTok{hist}\NormalTok{(sigma\_samples, }\AttributeTok{breaks =} \DecValTok{50}\NormalTok{, }\AttributeTok{freq=}\ConstantTok{FALSE}\NormalTok{, }\AttributeTok{main=}\StringTok{"Posterior Distribution of sigma"}\NormalTok{, }\AttributeTok{xlab=}\FunctionTok{expression}\NormalTok{(sigma), }\AttributeTok{col=}\NormalTok{color\_vector[}\DecValTok{5}\NormalTok{])}
\end{Highlighting}
\end{Shaded}

\includegraphics{Emanuele_Coradin_Rlab05_files/figure-latex/2.3-1.pdf}

\hypertarget{exercise-3-mcmc-inference-on-gaussian-model}{%
\section{Exercise 3 MCMC inference on Gaussian
model}\label{exercise-3-mcmc-inference-on-gaussian-model}}

\hypertarget{scenario-2}{%
\subsection{Scenario}\label{scenario-2}}

Suppose we observe the following values x = 2.06, 5.56, 7.93, 6.56, 2.05
and we assume that the data come from a Gaussian distribution with
unknown mean m and variance s2

\begin{itemize}
\item
  Build a simple JAGS model and run a Markov Chain Monte Carlo to obtain
  the posterior distribution of the mean and variance.
\item
  Assume uniform prior distributions for the parameters, m dunif(-10,
  10) and s dunif(0,50).
\item
  Compute also the posterior distribution for m/s.
\end{itemize}

\hypertarget{answers-2}{%
\subsection{Answers}\label{answers-2}}

\begin{Shaded}
\begin{Highlighting}[]
\CommentTok{\# Defining data}
\NormalTok{X }\OtherTok{\textless{}{-}} \FunctionTok{c}\NormalTok{( }\FloatTok{2.06}\NormalTok{, }\FloatTok{5.56}\NormalTok{, }\FloatTok{7.93}\NormalTok{, }\FloatTok{6.56}\NormalTok{, }\FloatTok{2.05}\NormalTok{)}

\NormalTok{N\_burnin }\OtherTok{\textless{}{-}} \DecValTok{1000} \CommentTok{\# length of the burn{-}in phase }
\NormalTok{thinning }\OtherTok{\textless{}{-}} \DecValTok{1}
\NormalTok{Nrep }\OtherTok{=} \DecValTok{10000}     \CommentTok{\# number of values to simulate}

\CommentTok{\# Let\textquotesingle{} define the model}
\NormalTok{model\_string }\OtherTok{\textless{}{-}} \StringTok{"model\{}
\StringTok{  }
\StringTok{  ratio \textless{}{-} mu/s2}
\StringTok{  }
\StringTok{  \# Likelihood}
\StringTok{  for (i in 1:length(X)) \{}
\StringTok{    X[i] \textasciitilde{} dnorm(mu, s2)}
\StringTok{  \}}

\StringTok{  \# Prior}
\StringTok{  mu \textasciitilde{} dunif({-}10, 10)}
\StringTok{  s2 \textasciitilde{} dunif(0, 50)}
\StringTok{  }
\StringTok{\}"}

\CommentTok{\# Compile jags model}
\NormalTok{dataList }\OtherTok{=} \FunctionTok{list}\NormalTok{(}\AttributeTok{X =}\NormalTok{ X)}
\NormalTok{model }\OtherTok{\textless{}{-}} \FunctionTok{jags.model}\NormalTok{(}\AttributeTok{file =} \FunctionTok{textConnection}\NormalTok{(model\_string), }
                    \AttributeTok{data =}\NormalTok{ dataList)}
\end{Highlighting}
\end{Shaded}

\begin{verbatim}
## Compiling model graph
##    Resolving undeclared variables
##    Allocating nodes
## Graph information:
##    Observed stochastic nodes: 5
##    Unobserved stochastic nodes: 2
##    Total graph size: 12
## 
## Initializing model
\end{verbatim}

\begin{Shaded}
\begin{Highlighting}[]
\CommentTok{\# Add burnin}
\FunctionTok{update}\NormalTok{(model, }\AttributeTok{n.iter =}\NormalTok{ N\_burnin)}

\CommentTok{\# Sample the posterior}
\NormalTok{posterior\_sample }\OtherTok{\textless{}{-}} \FunctionTok{coda.samples}\NormalTok{(model,}
                       \AttributeTok{variable.names =} \FunctionTok{c}\NormalTok{(}\StringTok{"mu"}\NormalTok{, }\StringTok{"s2"}\NormalTok{, }\StringTok{"ratio"}\NormalTok{),}
                       \AttributeTok{n.iter =}\NormalTok{ Nrep, }\AttributeTok{thin =}\NormalTok{ thinning)}
\FunctionTok{summary}\NormalTok{(posterior\_sample)}
\end{Highlighting}
\end{Shaded}

\begin{verbatim}
## 
## Iterations = 2001:12000
## Thinning interval = 1 
## Number of chains = 1 
## Sample size per chain = 10000 
## 
## 1. Empirical mean and standard deviation for each variable,
##    plus standard error of the mean:
## 
##          Mean      SD Naive SE Time-series SE
## mu     4.8340  1.1732 0.011732       0.016696
## ratio 33.3148 36.3621 0.363621       0.485388
## s2     0.2116  0.1204 0.001204       0.002047
## 
## 2. Quantiles for each variable:
## 
##          2.5%     25%     50%     75%    97.5%
## mu    2.47484  4.1563  4.8478  5.5230   7.1406
## ratio 8.76997 16.5341 24.4566 38.4228 114.0015
## s2    0.04506  0.1229  0.1889  0.2777   0.5075
\end{verbatim}

\begin{Shaded}
\begin{Highlighting}[]
\FunctionTok{plot}\NormalTok{(posterior\_sample)}
\end{Highlighting}
\end{Shaded}

\includegraphics{Emanuele_Coradin_Rlab05_files/figure-latex/3-1.pdf}

\begin{Shaded}
\begin{Highlighting}[]
\CommentTok{\# Retrieve the samples}
\NormalTok{posterior\_matrix }\OtherTok{\textless{}{-}} \FunctionTok{as.matrix}\NormalTok{(posterior\_sample)}

\NormalTok{mu\_samples }\OtherTok{\textless{}{-}}\NormalTok{ posterior\_matrix[, }\StringTok{"mu"}\NormalTok{]}
\NormalTok{s2\_samples }\OtherTok{\textless{}{-}}\NormalTok{ posterior\_matrix[, }\StringTok{"s2"}\NormalTok{]}
\NormalTok{ratio\_samples }\OtherTok{\textless{}{-}}\NormalTok{ posterior\_matrix[, }\StringTok{"ratio"}\NormalTok{]}

\CommentTok{\# Set up the plotting area to have 2 rows and 2 column}
\FunctionTok{par}\NormalTok{(}\AttributeTok{mfrow =} \FunctionTok{c}\NormalTok{(}\DecValTok{2}\NormalTok{, }\DecValTok{2}\NormalTok{))  }

\FunctionTok{acf}\NormalTok{(mu\_samples, }\AttributeTok{main =} \StringTok{"Autocorrelation of mu"}\NormalTok{)}
\FunctionTok{acf}\NormalTok{(s2\_samples, }\AttributeTok{main =} \StringTok{"Autocorrelation of s2"}\NormalTok{)}
\FunctionTok{acf}\NormalTok{(ratio\_samples, }\AttributeTok{main =} \StringTok{"Autocorrelation of ratio"}\NormalTok{)}
\end{Highlighting}
\end{Shaded}

\includegraphics{Emanuele_Coradin_Rlab05_files/figure-latex/3-2.pdf}

\begin{Shaded}
\begin{Highlighting}[]
\CommentTok{\# Computing the histograms and the step functions}

\NormalTok{mu\_hist }\OtherTok{\textless{}{-}} \FunctionTok{hist}\NormalTok{(mu\_samples, }\AttributeTok{breaks =} \DecValTok{50}\NormalTok{, }\AttributeTok{main =} \StringTok{"Posterior Distribution of mu"}\NormalTok{, }\AttributeTok{xlab =} \FunctionTok{expression}\NormalTok{(mu), }\AttributeTok{freq =} \ConstantTok{FALSE}\NormalTok{, }\AttributeTok{col =} \ConstantTok{NULL}\NormalTok{, }\AttributeTok{border =}\NormalTok{ color\_vector[}\DecValTok{6}\NormalTok{], }\AttributeTok{xlim=}\FunctionTok{c}\NormalTok{(}\SpecialCharTok{{-}}\DecValTok{2}\NormalTok{, }\DecValTok{12}\NormalTok{))}
\FunctionTok{plot\_intervals}\NormalTok{(mu\_hist)}
\end{Highlighting}
\end{Shaded}

\includegraphics{Emanuele_Coradin_Rlab05_files/figure-latex/3 plot-1.pdf}

\begin{verbatim}
## NULL
\end{verbatim}

\begin{Shaded}
\begin{Highlighting}[]
\NormalTok{s2\_hist }\OtherTok{\textless{}{-}} \FunctionTok{hist}\NormalTok{(s2\_samples, }\AttributeTok{breaks =} \DecValTok{50}\NormalTok{, }\AttributeTok{main =} \StringTok{"Posterior Distribution of s2"}\NormalTok{, }\AttributeTok{xlab =} \StringTok{"s2"}\NormalTok{, }\AttributeTok{freq =} \ConstantTok{FALSE}\NormalTok{, }\AttributeTok{col =} \ConstantTok{NULL}\NormalTok{, }\AttributeTok{border =}\NormalTok{ color\_vector[}\DecValTok{6}\NormalTok{])}
\FunctionTok{plot\_intervals}\NormalTok{(s2\_hist)}
\end{Highlighting}
\end{Shaded}

\includegraphics{Emanuele_Coradin_Rlab05_files/figure-latex/3 plot-2.pdf}

\begin{verbatim}
## NULL
\end{verbatim}

\begin{Shaded}
\begin{Highlighting}[]
\NormalTok{ratio\_hist }\OtherTok{\textless{}{-}} \FunctionTok{hist}\NormalTok{(ratio\_samples, }\AttributeTok{breaks =} \DecValTok{300}\NormalTok{, }\AttributeTok{main =} \StringTok{"Posterior Distribution of the ratio"}\NormalTok{, }\AttributeTok{xlab =} \StringTok{"ratio"}\NormalTok{, }\AttributeTok{freq =} \ConstantTok{FALSE}\NormalTok{, }\AttributeTok{col =} \ConstantTok{NULL}\NormalTok{, }\AttributeTok{border =}\NormalTok{ color\_vector[}\DecValTok{6}\NormalTok{], }\AttributeTok{xlim =} \FunctionTok{c}\NormalTok{(}\DecValTok{0}\NormalTok{,}\DecValTok{200}\NormalTok{))}
\FunctionTok{plot\_intervals}\NormalTok{(ratio\_hist)}
\end{Highlighting}
\end{Shaded}

\includegraphics{Emanuele_Coradin_Rlab05_files/figure-latex/3 plot-3.pdf}

\begin{verbatim}
## NULL
\end{verbatim}

\begin{Shaded}
\begin{Highlighting}[]
\CommentTok{\# Autocorrelation plot}
\NormalTok{lags }\OtherTok{\textless{}{-}} \FunctionTok{seq}\NormalTok{(}\DecValTok{0}\NormalTok{, }\DecValTok{150}\NormalTok{, }\DecValTok{5}\NormalTok{)}
\NormalTok{autocorrelation }\OtherTok{\textless{}{-}} \FunctionTok{autocorr}\NormalTok{(mcmc\_chain, }\AttributeTok{lags =}\NormalTok{ lags)}

\FunctionTok{plot}\NormalTok{(lags, autocorrelation, }\AttributeTok{type =} \StringTok{\textquotesingle{}s\textquotesingle{}}\NormalTok{, }\AttributeTok{col=}\NormalTok{color\_vector[}\DecValTok{7}\NormalTok{], }\AttributeTok{lty=}\DecValTok{1}\NormalTok{, }\AttributeTok{lwd=}\DecValTok{2}\NormalTok{, }\AttributeTok{main =} \StringTok{\textquotesingle{}Autocorrelation of the chain\textquotesingle{}}\NormalTok{, }\AttributeTok{xlab =} \StringTok{\textquotesingle{}lags\textquotesingle{}}\NormalTok{, }\AttributeTok{ylab=} \StringTok{\textquotesingle{}autocorrelation\textquotesingle{}}\NormalTok{) }\CommentTok{\#\textquotesingle{}log(|autocorrelation|)\textquotesingle{}}
\end{Highlighting}
\end{Shaded}

\includegraphics{Emanuele_Coradin_Rlab05_files/figure-latex/3 plot-4.pdf}

\hypertarget{exercise-4-mcmc-and-edwin-hubble-law}{%
\section{Exercise 4 MCMC and Edwin Hubble
law}\label{exercise-4-mcmc-and-edwin-hubble-law}}

\hypertarget{scenario-3}{%
\subsection{Scenario}\label{scenario-3}}

The data set that Edwin Hubble used to show that galaxies are moving
either away or towards us are given in the following table:

\begin{longtable}[]{@{}ll@{}}
\toprule\noalign{}
D {[}parsec{]} & V {[}km/s{]} \\
\midrule\noalign{}
\endhead
\bottomrule\noalign{}
\endlastfoot
0.0032 & 170 \\
0.0034 & 290 \\
0.2140 & -130 \\
0.2630 & -70 \\
0.2750 & -185 \\
------------ & --------- \\
0.2750 & -220 \\
0.4500 & 200 \\
0.5000 & 290 \\
0.5000 & 270 \\
0.6300 & 200 \\
------------ & --------- \\
0.8000 & 920 \\
0.9000 & 450 \\
0.9000 & 500 \\
0.9000 & 500 \\
0.9000 & 960 \\
------------ & --------- \\
2.0000 & 500 \\
2.0000 & 850 \\
2.0000 & 800 \\
2.0000 & 1090 \\
\end{longtable}

Using this data set define a JAGS model to fit data with the following:

\begin{itemize}
\item
  V{[}i{]} from dnorm(b * D{[}i{]}, c), where V represent the velocity
  in units of km/s, D is the observed distance (in units of parsec), and
  b and c are two parameters of the model
\item
  Assume whatever prior distribution you think is appropriate plot the
  evolution of the chains, the posterior distribution of the parameters
  and the 95\% credibility interval
\end{itemize}

\hypertarget{answers-3}{%
\subsection{Answers}\label{answers-3}}

\begin{Shaded}
\begin{Highlighting}[]
\CommentTok{\# Defining data}
\NormalTok{D }\OtherTok{\textless{}{-}} \FunctionTok{c}\NormalTok{( }\FloatTok{0.0032}\NormalTok{, }\FloatTok{0.0034}\NormalTok{, }\FloatTok{0.214}\NormalTok{,  }\FloatTok{0.263}\NormalTok{,  }\FloatTok{0.275}\NormalTok{,  }\FloatTok{0.275}\NormalTok{,  }\FloatTok{0.45}\NormalTok{,   }\FloatTok{0.5}\NormalTok{,    }\FloatTok{0.5}\NormalTok{,    }\FloatTok{0.63}\NormalTok{,   }\FloatTok{0.8}\NormalTok{,    }\FloatTok{0.9}\NormalTok{,    }\FloatTok{0.9}\NormalTok{,    }\FloatTok{0.9}\NormalTok{,    }\FloatTok{0.9}\NormalTok{,    }\DecValTok{2}\NormalTok{,  }\DecValTok{2}\NormalTok{,  }\DecValTok{2}\NormalTok{,  }\DecValTok{2}\NormalTok{)}

\NormalTok{V }\OtherTok{\textless{}{-}} \FunctionTok{c}\NormalTok{(}\DecValTok{170}\NormalTok{, }\DecValTok{290}\NormalTok{,    }\SpecialCharTok{{-}}\DecValTok{130}\NormalTok{,   }\SpecialCharTok{{-}}\DecValTok{70}\NormalTok{,    }\SpecialCharTok{{-}}\DecValTok{185}\NormalTok{,   }\SpecialCharTok{{-}}\DecValTok{220}\NormalTok{,   }\DecValTok{200}\NormalTok{,    }\DecValTok{290}\NormalTok{,    }\DecValTok{270}\NormalTok{,    }\DecValTok{200}\NormalTok{,    }\DecValTok{920}\NormalTok{,    }\DecValTok{450}\NormalTok{,    }\DecValTok{500}\NormalTok{,    }\DecValTok{500}\NormalTok{,    }\DecValTok{960}\NormalTok{,    }\DecValTok{500}\NormalTok{,    }\DecValTok{850}\NormalTok{,    }\DecValTok{800}\NormalTok{,    }\DecValTok{1090}\NormalTok{)}

\NormalTok{N\_burnin }\OtherTok{\textless{}{-}} \DecValTok{1000} \CommentTok{\# length of the burn{-}in phase }
\NormalTok{thinning }\OtherTok{\textless{}{-}} \DecValTok{1}
\NormalTok{Nrep }\OtherTok{=} \DecValTok{10000}     \CommentTok{\# number of values to simulate}

\CommentTok{\# Let\textquotesingle{} define the model}
\NormalTok{model\_string }\OtherTok{\textless{}{-}} \StringTok{"model\{}
\StringTok{  }
\StringTok{  \# Likelihood}
\StringTok{  for (i in 1:length(D)) \{}
\StringTok{    V[i] \textasciitilde{} dnorm(b*D[i], c)}
\StringTok{  \}}

\StringTok{  \# Prior}
\StringTok{  b \textasciitilde{} dunif(0, 1200)}
\StringTok{  c \textasciitilde{} dunif(0, 0.001)}
\StringTok{  }
\StringTok{\}"}

\CommentTok{\# Compile jags model}
\NormalTok{dataList }\OtherTok{=} \FunctionTok{list}\NormalTok{(}\AttributeTok{D =}\NormalTok{ D, }\AttributeTok{V=}\NormalTok{V)}
\NormalTok{model }\OtherTok{\textless{}{-}} \FunctionTok{jags.model}\NormalTok{(}\AttributeTok{file =} \FunctionTok{textConnection}\NormalTok{(model\_string), }
                    \AttributeTok{data =}\NormalTok{ dataList)}
\end{Highlighting}
\end{Shaded}

\begin{verbatim}
## Compiling model graph
##    Resolving undeclared variables
##    Allocating nodes
## Graph information:
##    Observed stochastic nodes: 19
##    Unobserved stochastic nodes: 2
##    Total graph size: 54
## 
## Initializing model
\end{verbatim}

\begin{Shaded}
\begin{Highlighting}[]
\CommentTok{\# Add burnin}
\FunctionTok{update}\NormalTok{(model, }\AttributeTok{n.iter =}\NormalTok{ N\_burnin)}

\CommentTok{\# Sample the posterior}
\NormalTok{posterior\_sample }\OtherTok{\textless{}{-}} \FunctionTok{coda.samples}\NormalTok{(model,}
                       \AttributeTok{variable.names =} \FunctionTok{c}\NormalTok{(}\StringTok{"b"}\NormalTok{, }\StringTok{"c"}\NormalTok{),}
                       \AttributeTok{n.iter =}\NormalTok{ Nrep, }\AttributeTok{thin =}\NormalTok{ thinning)}
\FunctionTok{summary}\NormalTok{(posterior\_sample)}
\end{Highlighting}
\end{Shaded}

\begin{verbatim}
## 
## Iterations = 2001:12000
## Thinning interval = 1 
## Number of chains = 1 
## Sample size per chain = 10000 
## 
## 1. Empirical mean and standard deviation for each variable,
##    plus standard error of the mean:
## 
##        Mean        SD  Naive SE Time-series SE
## b 4.578e+02 5.751e+01 5.751e-01      7.508e-01
## c 1.561e-05 4.900e-06 4.900e-08      6.688e-08
## 
## 2. Quantiles for each variable:
## 
##        2.5%       25%       50%       75%     97.5%
## b 3.436e+02 4.209e+02 4.574e+02 4.953e+02 5.715e+02
## c 7.455e-06 1.208e-05 1.513e-05 1.858e-05 2.647e-05
\end{verbatim}

\begin{Shaded}
\begin{Highlighting}[]
\FunctionTok{plot}\NormalTok{(posterior\_sample)}
\end{Highlighting}
\end{Shaded}

\includegraphics{Emanuele_Coradin_Rlab05_files/figure-latex/4-1.pdf}

\begin{Shaded}
\begin{Highlighting}[]
\CommentTok{\# Retrieve the samples}
\NormalTok{posterior\_matrix }\OtherTok{\textless{}{-}} \FunctionTok{as.matrix}\NormalTok{(posterior\_sample)}

\NormalTok{b\_samples }\OtherTok{\textless{}{-}}\NormalTok{ posterior\_matrix[, }\StringTok{"b"}\NormalTok{]}
\NormalTok{c\_samples }\OtherTok{\textless{}{-}}\NormalTok{ posterior\_matrix[, }\StringTok{"c"}\NormalTok{]}

\CommentTok{\# Set up the plotting area to have 1 rows and 2 column}
\FunctionTok{par}\NormalTok{(}\AttributeTok{mfrow =} \FunctionTok{c}\NormalTok{(}\DecValTok{1}\NormalTok{, }\DecValTok{2}\NormalTok{))  }

\FunctionTok{acf}\NormalTok{(b\_samples, }\AttributeTok{main =} \StringTok{"Autocorrelation of b"}\NormalTok{)}
\FunctionTok{acf}\NormalTok{(c\_samples, }\AttributeTok{main =} \StringTok{"Autocorrelation of c"}\NormalTok{)}
\end{Highlighting}
\end{Shaded}

\includegraphics{Emanuele_Coradin_Rlab05_files/figure-latex/4-2.pdf}

\begin{Shaded}
\begin{Highlighting}[]
\NormalTok{b\_hist }\OtherTok{\textless{}{-}} \FunctionTok{hist}\NormalTok{(b\_samples, }\AttributeTok{breaks =} \DecValTok{30}\NormalTok{, }\AttributeTok{main =} \StringTok{"Posterior Distribution of b"}\NormalTok{, }\AttributeTok{xlab =} \StringTok{"b"}\NormalTok{, }\AttributeTok{freq =} \ConstantTok{FALSE}\NormalTok{, }\AttributeTok{col =} \ConstantTok{NULL}\NormalTok{, }\AttributeTok{border =}\NormalTok{ color\_vector[}\DecValTok{6}\NormalTok{], }\AttributeTok{xlim =} \FunctionTok{c}\NormalTok{(}\DecValTok{200}\NormalTok{,}\DecValTok{800}\NormalTok{))}
\FunctionTok{plot\_intervals}\NormalTok{(b\_hist)}
\end{Highlighting}
\end{Shaded}

\includegraphics{Emanuele_Coradin_Rlab05_files/figure-latex/4 plot-1.pdf}

\begin{verbatim}
## NULL
\end{verbatim}

\begin{Shaded}
\begin{Highlighting}[]
\NormalTok{b\_best }\OtherTok{\textless{}{-}}\NormalTok{ b\_hist}\SpecialCharTok{$}\NormalTok{mids[ }\FunctionTok{which.max}\NormalTok{(b\_hist}\SpecialCharTok{$}\NormalTok{density) ]}


\NormalTok{c\_hist }\OtherTok{\textless{}{-}} \FunctionTok{hist}\NormalTok{(c\_samples, }\AttributeTok{breaks =} \DecValTok{30}\NormalTok{, }\AttributeTok{main =} \StringTok{"Posterior Distribution of c"}\NormalTok{, }\AttributeTok{xlab =} \StringTok{"c"}\NormalTok{, }\AttributeTok{freq =} \ConstantTok{FALSE}\NormalTok{,  }\AttributeTok{col =} \ConstantTok{NULL}\NormalTok{, }\AttributeTok{border =}\NormalTok{ color\_vector[}\DecValTok{6}\NormalTok{])}
\FunctionTok{plot\_intervals}\NormalTok{(c\_hist)}
\end{Highlighting}
\end{Shaded}

\includegraphics{Emanuele_Coradin_Rlab05_files/figure-latex/4 plot-2.pdf}

\begin{verbatim}
## NULL
\end{verbatim}

\begin{Shaded}
\begin{Highlighting}[]
\NormalTok{c\_best }\OtherTok{\textless{}{-}}\NormalTok{ c\_hist}\SpecialCharTok{$}\NormalTok{mids[ }\FunctionTok{which.max}\NormalTok{(c\_hist}\SpecialCharTok{$}\NormalTok{density) ]}


\FunctionTok{plot}\NormalTok{(D, V, }\AttributeTok{main=}\StringTok{"Fit Hubble Law"}\NormalTok{, }\AttributeTok{col=}\NormalTok{color\_vector[}\DecValTok{6}\NormalTok{], }\AttributeTok{pch=}\DecValTok{19}\NormalTok{)}
\FunctionTok{curve}\NormalTok{(b\_best}\SpecialCharTok{*}\NormalTok{x, }\AttributeTok{from =} \DecValTok{0}\NormalTok{, }\AttributeTok{to =} \DecValTok{13}\NormalTok{, }\AttributeTok{add=}\ConstantTok{TRUE}\NormalTok{, }\AttributeTok{col=}\NormalTok{color\_vector[}\DecValTok{7}\NormalTok{], }\AttributeTok{lwd=}\DecValTok{2}\NormalTok{)}
\end{Highlighting}
\end{Shaded}

\includegraphics{Emanuele_Coradin_Rlab05_files/figure-latex/4 plot-3.pdf}

\end{document}
